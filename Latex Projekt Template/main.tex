\documentclass[11pt]{article}

\usepackage[utf8]{inputenc}
\usepackage[ngerman]{babel}
\usepackage[top=2.5cm,bottom=2.5cm,left=1.7cm,right=1.7cm]{geometry}
\usepackage{amsmath}
\usepackage{csquotes}
\usepackage{fancyhdr}
\usepackage[iso,german]{isodate}
\usepackage{multirow}
\usepackage{xcolor}
\usepackage{graphicx}
\linespread{1.15}
\setlength{\headheight}{14pt}

\usepackage[style=numeric,backend=biber,defernumbers=true,]{biblatex}
\addbibresource{Quellen.bib} 
\usepackage{graphicx}
\graphicspath{{Abbildungen/}}

\usepackage[colorlinks=false,allbordercolors=white]{hyperref}

%Titelblatt - brauchen wir eigentlich erst im Endbericht:
\title{Template Studienarbeit}
\author{Sittner Christian}
\date{\today}

\newpage{}

\begin{document}

\pagestyle{fancy}

\fancyhf{}

\maketitle

\newpage

\renewcommand*\contentsname{Inhaltsverzeichnis}
\tableofcontents


%AB HIER RAUSKOPIEREN UND INS SAMMELDOKUMENT!!!!!!

\newpage
%Inhalt Kopfzeile - das wird im Enddokument nur noch einmal existieren:
\fancyhead[L]{A-02}
%Hier das Arbeitspaket für das ihr das Dokument erstellt:
\fancyhead[C]{APXX}
%Hier Version des Dokuments:
\fancyhead[R]{VXX}
%Hier der Name des Dokukmentes
\fancyfoot[L]{XXXXXXX}
\fancyfoot[C]{\thepage}
\fancyfoot[R]{\date{\today}}
%Headline-Tabelle:
\begin{tabular}{|p{3.5cm}|p{5cm}|p{2cm}|p{5cm}|}
 \hline
 Projekt &\multicolumn{3}{|l|}{Flügelanschluss Nutzflugzeug} \\
 \hline
 Projektphase &\multicolumn{3}{|l|}{Projektplanung} \\
 \hline
 Projektleiter &Professor Strohmaier &Ersteller &Sittner Christian\\
 &&&Weitere\\
 \hline
 AP-Veranwortlicher&Peter &Freigabe &Maggo\\
 \hline
\end{tabular}
%Dokumentbeginn:
\section{APXX Aufbau Section}

%BEI NEUEM PROJEKT: AB HIER KANN GELÖSCHT WERDEN!!!!!!

Wenn ihr ein Dokument erstellt, ist die Sektion euer Arbeitspaket (APXX)
\subsection{APXX.X und Subsection}
In die Subsection gehört das Unterarbeitspaket (APXX.X).
\subsubsection{Weitere Unterteilungen}
Ab hier ist die Aufteilung im Unterarbeitspaket euch überlassen
\subsubsection{Wichtig}
Keine Section darf nur einen Unterpunkt haben.\vspace{8pt}

\noindent Absätze müssen visuell von einander getrennt werden. Das Automatische Einrücken bitte nicht machen. Achtet drauf nicht zu lange Textabschnitte einzubauen. 
Es wäre außerdem schön im Code hin und wieder
mal
einen
Umbruch einzubauen, um Text leichter zu finden, sollten Sachen geändert werden müssen.\vspace{8pt}

\noindent Formeln könnt ihr auf zwei Arten in eurem Dokument hinterlegen. Mit Dollarzeichen $\frac{alles~hier~zwischen}{ist~eine~Formel}$ direkt im Fließtext.
Das nutzt ihr bitte nur für kurze Formelausdrücke, zum Beispiel wenn ihr die Kraft $F_D$ im Text erwähnt. Alle anderen Formeln sollten so programmiert werden:
%
%Formelformatierung:
\begin{equation}
 Die~Syntax~ist~dieselbe~wie~zwischen~den~Dollarzeichen:~\sum F_i    \to 50Nm\cdot 8=\int^2_1 dA=\ddot x=\dot v=a
\label{equation:APXX-01}%Das Label im Stil: "equation:APXX-"
\end{equation}
% 
Das Label 'equation:APXX-' (\ref{equation:APXX-01}) bitte einhalten, damit wir beim Zusammenfügen nicht 20 verschiedene Gleichungen 01 haben.
%
\noindent Bilder müssen im Text referenziert werden (siehe Abbildung \ref{fig:Beispielbild})\vspace{8pt}

\noindent Quellen müssen zitiert werden. Die Quellen sind in der Datei 'Quellen.bib' hinterlegt. Es gibt unterschiedliche Arten von Quellen. 
Websiten \cite{website}, Dokumente die wir im Rahmen des Projekts erstellt haben \cite{mpp}, (diese müssen dann auch noch in den Anhang), vom Strohmeier erzählt \cite{strohmeier},
Skripte und Themen aus anderen Vorlesungen \cite{aero}. Ich hab jetzt in den Lastannahmen noch Pietras \cite{bdlII} und die Berichte vom Sperl \cite{b236B} zu Rate 
gezogen. Es wäre ne Überlegung Wert ein Sammeldokument für alle Quellen im Root der Datenbank anzulegen, damit wir nicht unterschiedlich dieselben Quellen zitieren.\vspace{8pt}
\newpage
%
%Bilder im Pfad /figures
\begin{figure}[h]
    \centering
    \includegraphics[width=0.75\textwidth]{Beispielbild} 
    \caption{Die Caption darf auch nicht leer bleiben!}
    \label{fig:Beispielbild}%Das Label bitte auch eindeutig vergeben!
\end{figure}
%
%
\noindent Wenn ein Bild von Latex auf die nächste Seite verschoben wird, mag ich es die Codezeilen dahin zu setzen wo auch das Bild hingeschoben wird. 
Erleichtert die Syntax. Bilder sucht der Code im Ordner 'Abbildungen'. Das '.png' braucht ihr nicht zwingend am Ende.\vspace{8pt}

\noindent Es ist übrigens auch möglich Textzeilen zu zitieren:\vspace{8pt}

\noindent Zitat \cite{cs}
\begin{quotation}
\begin{itshape}
\noindent "Positive and negative gusts of 25 fps at VD must be considered at altitudes between
sea level and 6096 m (20 000 ft). The gust velocity may be reduced linearly from
25 fps at 6096 m (20 000 ft) to $12.5$ fps at 15240 m (50 000 ft)."
\end{itshape}
\end{quotation}\vspace{8pt}
\noindent Die Formelsyntax kann man recht schnell lernen, aber es gibt online Formeleditoren, die helfen am Anfang \cite{formel}, Sonderzeichen google ich immer. \vspace{8pt}

\noindent Ich glaube jetzt ist alles wichtige drin. Am besten immer das Template bei nem neuen Projekt umbauen, ich makier die Zeilen raus gelöscht werden müssen.
Ich benutze Latex über VS-Code mit lokalem Compiler, das Setup ist recht simpel, das Tutorial \cite{latex} dauert 15 min. Overleaf hat sehr gute Tutorials, auch wenn man nur
Syntax sucht.\vspace{8pt}

\noindent Wenn ihr ein Latexprojekt den Server hochladet müsst Ihr in Dokumentoriginale 'main.tex', 'Quellen.bib' 
und den Ordner 'Abbildungen' hochladen.  Das pdf solltet ihr auch beilassen, damit man nicht jedes Mal compilen muss um das Dokument zum Code zu sehen. 
BITTE ein Projekt immer in einen eigenen, beschrifteten Ordner tun - sonst haben wir bald 
20 'main.tex' Dateien die nicht zusammenhängen.

%BEI NEUEM PROJEKT: BIS HIER KANN GELÖSCHT WERDEN!!!!!!

%BIS HIER RAUSKOPIEREN UND INS SAMMELDOKUMENT!!!!!!

\newpage
\section{Literaturverzeichnis}
\printbibliography[heading=subbibintoc,type=report,title={Referenzdokumente, im Rahmen des Projekts erstellt}]
\printbibliography[heading=subbibintoc,type=reference,title={Skript-Quellen}]
\printbibliography[heading=subbibintoc,type=online,title={Online-Quellen}]
\end{document}
